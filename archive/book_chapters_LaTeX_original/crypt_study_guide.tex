\section{Study guide  for ``Applications (I): Introduction to Cryptography''  chapter}
\label{sec:Cryptography:StudyGuide} 

\subsection*{Section \ref{sec:Cryptography:PrivateKey}, Private key cryptography}
\subsubsection*{Concepts:}
\begin{enumerate}
\item 
Shift codes (monoalphabetic cryptosystem -- one-to-one substitution)
\item
Affine codes (monoalphabetic cryptosystem -- one-to-one substitution)
\item
Affine codes (polyalphabetic cryptosystem -- ciphertext represents more than one letter)
\item
Modular matrix multiplication
\item
Matrix inverses in ${\mathbb Z}_{n}$
\end{enumerate}

\subsubsection*{Competencies}
\begin{enumerate}
\item
Know how to encode and decode using the shift code method.   (\ref{exercise:Cryptography:encode1}, \ref{exercise:Cryptography:decode1}, \ref{exercise:Cryptography:plaintext1}, \ref{exercise:Cryptography:plaintext2})
\item
Be able to find the decoding function when given a valid encoding affine function.  (\ref{exercise:Cryptography:affine1}, \ref{exercise:Cryptography:affine2})
\item
Be able to solve modular matrix multiplication.  (\ref{exercise:Cryptography:mod_mult})
\item
Be able to find matrix inverses in ${\mathbb Z}_{n}$, when they exist.  (\ref{exercise:Cryptography:minv})
\end{enumerate}


\subsection*{Section \ref{sec:Cryptography:PublicKey}, Public key cryptography}
\subsubsection*{Concepts:}
\begin{enumerate}
\item 
RSA cryptosystem (more advanced encryption system: uses modular exponentiation to encrypt and decrypt messages)
\item
Binary expansion (like decimal expansion, except it uses base 2 instead of base 10)
\item
Identifying prime numbers by brute force (Euler totient function and sieve of Eratosthenes)
\item
Identifying prime numbers by Fermat's test for primality (Fermat's factorization algorithm)
\item
Pseudoprime numbers
\end{enumerate}

\subsubsection*{Key formulas}
\begin{enumerate}
\item
Fermat's factorization algorithm: If $n$ is an odd composite number, then $n = x^{2} - y^{2} = (x - y)(x + y)$ for some $x$ and $y$
\item
Pseudoprime formula: the odd number $n$ is a pseudoprime base $b$ if mod$(b^{n-1},n) = 1$
\end{enumerate}

\subsubsection*{Competencies}
\begin{enumerate}
\item
Compute binary expansion of exponent, either by hand (\ref{exercise:Cryptography:power}) or by spreadsheet (\ref{exercise:Cryptography:powerplus}).
\item
Using binary expansion of exponent to rapidly compute modular exponentials by spreadsheet. (\ref{exercise:Cryptography:power}, \ref{exercise:Cryptography:powerplus})
\item
Given a base, encoding (decoding) key, and message, encrypt (decrypt) RSA messages. (\ref{exercise:Cryptography:RSA_E}, \ref{exercise:Cryptography:RSA_D})
\item
Given a base and encoding (or decoding) key, use brute force method by spreadsheet to find the corresponding decoding (or encoding) key. (\ref{exercise:Cryptography:brute})
\item
Use Fermat's factoring method by spreadsheet to factor large numbers. (\ref{exercise:Cryptography:FermatSpreadsheet})
\item
Determine if a number is pseudoprime relative to a given base. (\ref{exercise:Cryptography:prime_pseudo})
\end{enumerate}
