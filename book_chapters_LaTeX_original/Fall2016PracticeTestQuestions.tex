\chap{Example Test Questions}{PracticeTestQuestions_chap}

The following exercises were taken from various exams. The list will be added to in subsequent editions of this text.  Contributions are welcome.


\section{Preliminaries}
\begin{enumerate}[(1)]
\item
Simplify:
$\displaystyle{ \left(\frac{(x+y)^{x+y}(x-y)^{x-y}}{(x^2 - y^2)^x}\right)}$
\item
Simplify:
$ \displaystyle{\left( \frac{6^6}{2^2 3^3} +  \frac{2^8 3^6}{6^3}\right)^{1/2}} $
%\item
%Simplify:   $ \displaystyle{\frac{(a+b)(b+c) + (a-b)(b-c)}{a+c} }$
%%%Actual
%\item
%Simplify:   $ \displaystyle{\frac{(a+b)(b-c) + (a-b)(b+c)}{c-a} }$
\item
Given the expression:  $( a(bc - cb) + (ac - ca)b) + c(ab - ba)$:
\begin{enumerate}[(a)]
\item
Simplify, using the associative property ONLY.
\item
Simplify, using the associative and distributive properties ONLY.
\item
Simplify, using associative, distributive, and commutative properties.
\end{enumerate}

\item
Given the expression:~
 $(((a-b)+b)+b)(a-b) + b^2$
\begin{enumerate}[(a)]
\item
Simplify the expression using the associative law ONLY.
\item
Simplify the expression using the associative and distributive laws ONLY.
\item
Simplify the expression using the associative, distributive, and commutative laws.
\end{enumerate}

%\item
%Given the expression:~
% $(p-q)(p+q) + (q-p)(p+q)$
%\begin{enumerate}[(a)]
%\item
%Simplify the expression using the associative law ONLY.
%\item
%Simplify the expression using the associative and distributive laws ONLY.
%\item
%Simplify the expression using the associative, distributive, and commutative laws.
%\end{enumerate}

%%% Actual
%\item
%Given the expression:~
% $(r+p)(s+q) - (p+s)(q+r)$
%\begin{enumerate}[(a)]
%\item
%Simplify the expression using the associative law ONLY.
%\item
%Simplify the expression using the associative and distributive laws ONLY.
%\item
%Simplify the expression using the associative, distributive, and commutative laws.
%\end{enumerate}


\item
Give an example (using actual numbers) to show that division is not associative.
\item
Suppose $ab>cb, b < 0,$ and $c<0$.  For each of the following statements, either prove that it is always true, or give an example to show that
it is not always true:
\begin{enumerate}[(a)]
\item
$a > b$ \qquad 
\item
$a < 0$.
\item
$b < c$ \qquad 
\item
$a < c$ \qquad 
\end{enumerate}

\end{enumerate}

\section{Complex Numbers}

\begin{enumerate}

\item
Evaluate: $(\sqrt{6}+3\sqrt{2}i)^6/36$.

\item
Prove that  $i(z + \bar{z})(z - \bar{z})$ is real for any complex number $z$

\item
Suppose that $z$ is a complex number such that $z^{-1} = \bar{z}$.
\begin{enumerate}[(a)]
\item
 Find the modulus of $z$. (\emph{Hint}:: Use polar form.)
\item
How many solutions does this equation have?
\end{enumerate}
	
\item
Find all solutions to:  $\displaystyle{z^{-3} = 8i.}$

\item
Find all solutions to:  $z^5 = -32i.$

%%% Actual
%\item
%Find all solutions to: $z^4 = -16$.

\item
Evaluate:  $\displaystyle{ \left(\overline{3 + 8i} \right)  \, / \, \left(7 + 6i \right)}$.
\item
Evaluate:  $\displaystyle{( \overline{4 -7i} ) \cdot (\overline{3 + 3i})^{-1}}$.
\item
$z$ and $w$ are complex numbers. $z$ has modulus 7 and complex argument $\pi/9$, while $w$ has modulus $\sqrt{7}$ and argument $\pi/6$.  What are the modulus and argument of $z^3 w^{-4}$?

\item
Evaluate $\left(\frac{1-i}{2}\right)^{10}$.  Show your work. Give your answer in the form $a + bi$, with no decimals.

\item
Evaluate $ i^{2x+3}$, where $x=12345^{12345}$.

%\item
%It is possible to raise numbers to imaginary powers.  A famous formula of Euler states that   
%$e^{i\pi}= -1$, Where $e$ is the mathematical constant  $2.71828 \ldots$ .  Use this expression to show that $i^i$  is a positive real number between 0 and 1.  (\emph{Hint}: take the square root of both sides of Euler's formula.)

\item
Simplify:  $(z + \bar{z})(z - \bar{z}) + \overline{(z + \bar{z})(z - \bar{z})}$. \emph{Show your work.}

%\item
%A cubic polynomial of the form $x^3 + ax^2 + bx + c$  ($a,b,c$ are real)  has roots $11$ and $3-i$.  Find $a,b,c$.
%
%\item
%A polynomial of the form $x^4 + a_3x^3 + a_2x^2 + a_1x + a_0$  ($a_0,a_1,a_2,a_3$ are real)  has roots $3+2i$ and $1-i$.  Find $a_0,a_1,a_2,a_3$.

\item
Find all $6^{\text{th}}$ roots of  $8i$.


\item
Find all fifth roots of $-1-i$.

\item
Suppose $\bar{z} = iz$ and $z \neq 0$. Show that $z^4$ is a negative real number.

\item
Draw a picture of the following set in the complex plane:  $|\textrm{Re}[z] | = 2$.  (Recall that $\textrm{Re}[z]$ means the real part of $z$.)

%\item
%Show that there exists a complex number $z$ such that $\sin(t) = \text{Re}[z \cdot \cis(t)]$, and find $z$.
%
%\item
%\begin{enumerate}[(a)]
%\item 
%Find a complex number $w$ such that $\sin(t) - \cos(t) = \textrm{Re}[ w \cdot \cis(t) ]$. Express $w$ in complex polar form.  
%\item
%Using part (a), show that  $\sin(t) - \cos(t)$ can be written as $ A \cdot \cos(t + \theta)$. Find $A$ and $\theta$.  
%\end{enumerate}

\end{enumerate}



\section{Modular Arithmetic}

\begin{enumerate}

\item
December 25, 2015 is on a Friday.  What day of the week is December 1, 2018?  (Note: 2016 has 366 days, and other years have 365 days).

\item
A certain computer program takes 4,923 hours to run to completion. The program is started on Monday at 9:00 a.m.
\begin{enumerate}[(a)]
\item
What is the time on the clock when the program completes?
\item
On what day of the week does the program complete?
\end{enumerate}



\item
Make tables that show the results of:
\begin{enumerate}[(a)]
\item \label{Mod3TablesEx-multiplication}
multiplication modulo~$4$.
\item \label{Mod3TablesEx-subtraction}
addition modulo~$4$
\end{enumerate}
%For both (a) and (b), all table entries should be  $\class{0} \ldots \class{3}$.



%\item
%Find integers $m$ and $n$ that solve the following equation: $4801m + 500n = 1337$.

\item
Solve these simultaneous congruences: $x \equiv 1 \bmod{2}; x \equiv 2 \bmod{3}; x \equiv 3 \bmod{5}; x\equiv4 \bmod{7}$.


%\item
%Show that  modular multiplication distributes over modular addition: That is, given $x,y,z \in \integer_n$ we have
%\[
%x \odot (y \oplus z) = (x \odot y) \oplus  (x \odot z).
%\]
%(\emph{Hint}:  First apply Exercise~\ref{exercise:modular:ops} part (c)  with $e=f=x, a=b=y, c=d=z$. Then, use the ordinary distributive law. on the left side of the modular equivalence.  Then, use  Proposition~\ref{proposition:modular:number_remainder} parts (a) and (b) to return to an expression with modular operations.  Finally, use Proposition~\ref{proposition:modular:equiv_mod_n} to obtain an equality instead of modular equivalence.)

%\item
% Find a solution to:  $411m + 312n = 41 $.

\item
Find all solutions to: $277x \equiv 149 \pmod{113}$.

\item
Find all solutions to:  $228 x - 104 \equiv 777 \pmod{56}$


\item
Find all solutions to:  $ 470x - 120 \equiv 852 \pmod{93}$

\item
Compute:  mod($30!,19$).  (Note: $30!$ means $1 \cdot 2 \cdot 3 \cdot \ldots \cdot 30$.)

\item
Compute mod( $3^{100}$,5). 


\item
Evaluate: 
\begin{enumerate}[(a)]
\item
 gcd(111,507) 
\item
gcd(182,367) 
\item
gcd(39,409)
\end{enumerate}

\item
Find all values of $m$ and $n$ that solve the following equations:
\begin{enumerate}[(a)]
\item
$88m + 97n = 19$
\item
 411m + 312n = 41 
\item
105m + 75n = 225
%\item
%497m + 224n = 35
\end{enumerate}


\item
\begin{enumerate}[(a)]
\item
Show that if $\bmod(x,4)=1$, then $\bmod(x^2,8)=1$.
\item
Show that if $\bmod(x,4)=2$, then $\bmod(x^2,8) = 4$.
\item
Show that if $\bmod(x,4)=3$ then $\bmod(x^2,8) = 1$.
\item	
Show that if $x,y$  are integers such that $\bmod(x^2+y^2,8)=2$, then both $x$ and $y$ must be odd.
\item	
Show that if $x,y,z$ satisfy $x^2 + y^2 + 6 = 8z$, then $x$ and $y$ must both be odd.
\end{enumerate}

%%% Actual
%\item
%\begin{enumerate}[(a)]
%\item
%Show that if $\bmod(x,3)=1$, then $\bmod(x^2,3)=1$.
%\item
%Show that if $\bmod(x,3)=2$, then $\bmod(x^2,3) = 1$.
%\item
%Show that if $\bmod(x,3)=0$ then $\bmod(x^2,3) = 0$.
%\item	
%Show that if $x,y,z$  are integers such that $x^2 + y^2 = 3z+1$ , then either $x$ or $y$ is divisible by 3.
%\end{enumerate}


%\item
%\begin{enumerate}[(a)]	
%\item
%Show that if $\bmod(x,4)=1$, then $\bmod(x^4,16)=1$.
%\item	
%Show that if $\bmod(x,4)=2$ then $\bmod(x^4,16) = 0$.
%\item	
%Show that if $\bmod(x,4)=3$ then $\bmod(x^4,16) = 1$.
%\item	
%Show that if $\bmod(x,4)=0$ then $\bmod(x^4,16) = 0$.
%\item
%Show that if $\bmod(x^4+y^4,16) = 0$ then $x$ and $y$ must both be even.
%\item	
%Show that if $x,y, z$ are integers such that $x^4+y^4 = 16z$, then $x$ and $y$ must both be even.
%\end{enumerate}

\item
Find $x,y \in \integer_{51}$ such that $x \neq 0$ and $y \neq 0$, but $x \odot y = 0$.


\end{enumerate}


\section{Set Theory}

\begin{enumerate}

\item
Let ${\mathbb N}$ be the universal set and suppose that
\begin{align*}
A &= \{ x \in {\mathbb N} : x \text{ is a perfect square (that is, } x=y^2 \text{ where } y \text{ is a natural number)}\} \\ 
B &= \{ x \in {\mathbb N} : x \text{ is divisible by 6}\} \\ 
C &= \{ x \in {\mathbb N} : x  \equiv 2 \text{ (mod 3)} \\
D &= \{ x \in {\mathbb N} : x  \equiv 0 \text{ (mod 3)} \\
\end{align*} 
Specify each of the following sets. You may specify a set either by describing a property, by enumerating the elements, or as one of the four sets $A, B, C, D$:
\begin{enumerate}[(a)]
\item
$(A \cap B)$
\item
$B \cap C$
\item
$C \cup (B \setminus D)$.
\end{enumerate}





\end{enumerate}

