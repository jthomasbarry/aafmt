
\section{Solutions for ``Set Theory''}
\noindent\textbf{\textit{ (Chapter \ref{sets})}}\bigskip

\textbf{Exercise \ref{exercise:sets:set1}:}\\
d. i) \{1,3,5,...\}\\
d. ii) \{$x$ : $x$ is not divisbile by 2\}\\
\\
\textbf{Exercise \ref{exercise:sets:s16}:}\\
a. 2\\
b. 1\\
c. 1\\
d. 0\\
\\
\textbf{Exercise \ref{exercise:sets:7}:}\\
a. No; A set with 2 elements\\
b. 4 elements\\
\\
\textbf{Exercise \ref{exercise:sets:12}:}\\
b. Let $A_1=\{ 1+i,2+2i\}$, $A_2=\{ 3+3i,4+4i\}$, and $A_3=\{5+5i,6+6i\}$ then we have:
$A_1,A_2,A_3 \subset C$, $A_1 \cap A_2=\emptyset$, $A_1 \cap A_3=\emptyset$, $A_2 \cap A_3=\emptyset$, and $A_1 \cap A_2 \cap A_3=\emptyset$.\\
\\
\textbf{Exercise \ref{exercise:sets:14}:}\\
a.$\bigcap_{i = 1}^{n}  \{i\}=\{\}$\\
\\
b. $\bigcap^{n}_{i=1} \{ 1,2,...,i \}=\{1\}$\\
\\
c. $\bigcap^{\infty}_{i=1} \{ 1,2,...,i \}=\{1\}$\\
\\
e. $\bigcap_{r = 0}^{n-1}  \{\mbox{Integers that have remainder }r \mbox{ when divided by }n\}=\{\}$\\
\\
\textbf{Exercise \ref{exercise:sets:15}:}\\
a. $A_i=\{x\in R|i-1 \le x\le i\}$ with $i=1,2,3,...$\\
b. $A_i=\{x\in R|i-1 < x\le i\}$ with $i=1,2,3,...$\\
\\
\textbf{Exercise \ref{exercise:sets:16}:}\\
a. $A_1=\{...,-3,-2,-1\} $\\
$A_2=\{0\}$\\
$A_3=\{1,2,3\}$\\
$A_4=\{4,5,6,7,...\}$\\
b. $A_1=\{x\in R|x<0\}$\\
$A_2=\{0\}$\\
$A_3=\{x\in R|0<x<1\}$\\
$A_4=\{x\in R|x\ge 1\}$\\
\\
\textbf{Exercise \ref{exercise:sets:18}:}\\
$A\subset B$. The largest subset of $B$ that is disjoint from $A$ is $B\setminus A$\\
\\
\textbf{Exercise \ref{exercise:sets:20}:}\\
a. $(A\cap B)\setminus C=\emptyset$\\
b. $A\cap B\cap C\cap D=\emptyset$\\
c. $A \cup B \cup C \cup D$=$\mathbb{N}$\\
\\
\textbf{Exercise \ref{exercise:sets:21}:}\\
a. $A\cap B=\{2\}$\\
c. $A'\cap B'=\{x\in N|$ x is odd and x is not prime$\}$\\
e. $(A\cup B)'=\{x\in N|$ x is odd and x is not prime$\}$\\
f. $A'\cup B'=\{x\in N|$ x is odd or x is not prime$\}$\\
\\
\textbf{Exercise \ref{exercise:sets:23}:}\\
First, suppose that $x$ is an element of $A$, then we have:\\
$x\in A$ (supposition) $\implies x\in A$ or $x\in B$ (logic) $\implies x\in A\cup B$ (def. of union)\\
Since every element of A is an element of $A\cup B$, it follows by the definition of subset that $A\subset (A\cup B)$\\
\\
\textbf{Exercise \ref{exercise:sets:26}:}\\
(Part 6 of Prop. 24)\\
a. Suppose $x$ is an element of $A\cup B$\\
$\implies x\in A\cup B$ (supposition)\\
$\implies x\in A$ or $x\in B$ (def. of union)\\
$\implies x\in B$ or $x\in A$ (logic)\\
$\implies x\in B\cup A$ (def. of union)\\
Since every element of $A\cup B$ is an element of $B\cup A$, we have $(A\cup B)\subset(B\cup A)$ (1)\\
Similarly, we can prove every element of $B\cup A$ is an element of $A\cup B$, so we have $(B\cup A)\subset (A\cup B)$ (2)\\
From (1) and (2) $\implies A\cup B = B\cup A$\\
b. Similarly, steps by steps as in part (a), we can prove that $A\cap B=B\cap A$\\
\\
\textbf{Exercise \ref{exercise:sets:s27}:}\\
Assume $x$ is in $(A\cup B)'$\\
Then $x \notin A \cup B)$ [def. complement]\\
Then $x \notin A$ or $B$ [def. union]\\
Then $x \notin A$ and $x \notin B$ [logic]\\
Then $x \in A'$ and $x \in B'$ [def. complement]\\
Then $x \in A' \cap B'$ [def. interection]\\
\\
\textbf{Exercise \ref{exercise:sets:30}:}\\
a.\begin{align*}
(A\cap B) \setminus B & =(A\cap B)\cap B'\\
& = A\cap (B\cap B')\\
& = A\cap \emptyset\\
& = \emptyset\\
\end{align*}
b.\begin{align*}
(A\cup B) \setminus B & =(A\cup B)\cap B'\\
& = B'\cap (A\cup B)\\
& = (B'\cap A)\cup(B'\cap B)\\
& = (B'\cap A)\cup \emptyset\\
& = B'\cap A\\
& = A\cap B'\\
& = A\setminus B\\
\end{align*}
c.\begin{align*}
A\setminus (B\cup C) & =A\cap (B\cup C)'\\
& = A\cap (B'\cap C')\\
& = A\cap A\cap(B'\cap C')\\
& = (A\cap B')\cap (A\cap C')\\
& = (A\setminus B)\cap (A\setminus C)\\
\end{align*}
\textbf{Exercise \ref{exercise:sets:31}:}\\
a. $S=\{a,b,c\}$\\
Subsets of S: $\{a\};\{b\};\{c\};\{a,b\};\{b,c\};\{c,a\};\{a,b,c\};\emptyset$\\
There are 8 subsets.\\
b. $S=\{a,b\}$. Similarly, there are 4 subsets of S.\\
c. $S=\{a,b,c,d\}$. There are 16 subsets of S as follows:\\
$\{a\};\{b\};\{c\};\{d\}$\\
$\{a,b\};\{a,c\};\{a,d\};\{b,c\};\{b,d\};\{c,d\}$\\
$\{a,b,c\};\{a,b,d\};\{b,c,d\};\{a,c,d\}$\\
$\{a,b,c,d\};\emptyset$\\
\\
\textbf{Exercise \ref{exercise:sets:cup_group}:}\\
$G$ is the set of subsets of the set $\{a,b,c\}$\\
a. $G$ is closed under the $\cup$ operation. Union of any element in $G$ will be still in $G$.\\
b. The identity is $\emptyset$. We have $A\cup \emptyset=A$.\\
c. The operation $\cup$ is associative.\\
d. The operation $\cup$ is commutative.\\
e. No element of $G$ has a unique inverse under the operation $\cup$.\\
For example: $\{a\}$ has no inverse $I$ to make $\{a\}\cup I=\emptyset$\\
f. So $G$ is not a group under the operation $\cup$ because elements in $G$ have no inverse.\\