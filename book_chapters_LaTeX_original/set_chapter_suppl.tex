
\section{Solutions for ``Set Theory''}
\label{sec:AnswerKey:Sets}
\noindent\textbf{\textit{ (Chapter \ref{Sets})}}\bigskip

\noindent\textbf{Exercise \ref{exercise:Sets:set1}:}\\
d. i) \{1,3,5,...\}\\
d. ii) \{$x$ : $x$ is not divisbile by 2\}\\

\noindent\textbf{Exercise \ref{exercise:Sets:SetSet}:}\\

\noindent\textbf{Exercise \ref{exercise:Sets:s16}:}\\
a. 2\\
b. 1\\
c. 1\\
d. 0\\

\noindent\textbf{Exercise \ref{exercise:Sets:7}:}\\
a. No; A set with 2 elements\\
b. 4 elements\\

%\noindent\textbf{Exercise \ref{exercise:Sets:}:}\\ %KW unnamed exercise

\noindent\textbf{Exercise \ref{exercise:Sets:12}:}\\
b. Let $A_1=\{ 1+i,2+2i\}$, $A_2=\{ 3+3i,4+4i\}$, and $A_3=\{5+5i,6+6i\}$ then we have:
$A_1,A_2,A_3 \subset C$, $A_1 \cap A_2=\emptyset$, $A_1 \cap A_3=\emptyset$, $A_2 \cap A_3=\emptyset$, and $A_1 \cap A_2 \cap A_3=\emptyset$.\\

\noindent\textbf{Exercise \ref{exercise:Sets:14}:}\\
a.$\bigcap_{i = 1}^{n}  \{i\}=\{\}$\\
\\
b. $\bigcap^{n}_{i=1} \{ 1,2,...,i \}=\{1\}$\\
\\
c. $\bigcap^{\infty}_{i=1} \{ 1,2,...,i \}=\{1\}$\\
\\
e. $\bigcap_{r = 0}^{n-1}  \{\mbox{Integers that have remainder }r \mbox{ when divided by }n\}=\{\}$\\

\noindent\textbf{Exercise \ref{exercise:Sets:15}:} %KW updated b, c,d
\begin{enumerate}[(a)]
\item
%Find an infinite collection of sets $\{A_i\}, i = 1,2,3,\ldots$ such that (i) $A_i \subset {\mathbb R}, i = 1,2,3, \dots$; (ii) each $A_i$ is a closed  interval of length 1 (that is, $A_i = [a_i, a_i+1]$ for some $a_i$; and (iii) $\bigcup_{i = 1}^{\infty} A_{i} = [0, \infty)$. (That is, the union of all the $A_i$'s is the set of all nonnegative real numbers.)
$A_i=\{x\in R|i-1 \le x\le i\}$ with $i=1,2,3,...$

\item
%Find an infinite collection of sets $\{A_i\}, i = 1,2,3,\ldots$ such that (i) $A_i \subset {\mathbb R}, i = 1,2,3, \dots$; (ii) each $A_i$ is an open  interval of length 1 (that is, $A_i = (a_i, a_i+1)$ for some $a_i$; and (iii) $\bigcup_{i = 1}^{\infty} A_{i} = (0, \infty)$. (That is, the union of all the $A_i$'s is the set of all positive real numbers.)
$A_i=\{x\in R|i-1 < x < i\}$ with $i=1,2,3,...$

\item
%Find an infinite collection of sets $\{A_n\}, n = 1,2,3,\ldots$ such that (i) $A_n \subset  [-1/2,1/2], n = 1,2,3, \dots$; (ii)  each $A_n$ is an open interval of length $1/n$; and (iii) $\bigcap_{n = 1}^{\infty} A_{n} = \{0\}$.
$A_{n}=\{x\in \mathbb {R}|-\frac{1}{2n}< x < \frac{1}{2n}\}$ with $n=1,2,3,...$

 \item
%**Find an infinite collection of sets $\{A_n\}, n = 1,2,3,\ldots$ such that (i) $A_n \subset [0,1], n = 1,2,3, \dots$; (ii) each $A_n$ is an open interval of length $1/n$; (iii) $A_{n+1} \subset A_{n}, n = 1,2,3, \dots$; and (iv) $\bigcap_{n = 1}^{\infty} A_{n} = \emptyset$.
$A_{n}=\{x\in \mathbb {R}|0 < x < \frac{1}{n}\}$ with $n=1,2,3,...$
\end{enumerate}

\noindent\textbf{Exercise \ref{exercise:Sets:16}:}\\
a. $A_1=\{...,-3,-2,-1\} $\\
$A_2=\{0\}$\\
$A_3=\{1,2,3\}$\\
$A_4=\{4,5,6,7,...\}$\\
b. $A_1=\{x\in R|x<0\}$\\
$A_2=\{0\}$\\
$A_3=\{x\in R|0<x<1\}$\\
$A_4=\{x\in R|x\ge 1\}$\\

\noindent\textbf{Exercise \ref{exercise:Sets:18}:}\\
$A\subset B$. The largest subset of $B$ that is disjoint from $A$ is $B\setminus A$\\

\noindent\textbf{Exercise \ref{exercise:Sets:20}:}\\
a. $(A\cap B)\setminus C=\emptyset$\\
b. $A\cap B\cap C\cap D=\emptyset$\\
c. $A \cup B \cup C \cup D$=$\mathbb{N}$\\

\noindent\textbf{Exercise \ref{exercise:Sets:21}:}\\
a. $A\cap B=\{2\}$\\
c. $A'\cap B'=\{x\in N|$ x is odd and x is not prime$\}$\\
e. $(A\cup B)'=\{x\in N|$ x is odd and x is not prime$\}$\\
f. $A'\cup B'=\{x\in N|$ x is odd or x is not prime$\}$\\

\noindent\textbf{Exercise \ref{exercise:Sets:23}:}\\
First, suppose that $x$ is an element of $A$, then we have:\\
$x\in A$ (supposition) $\implies x\in A$ or $x\in B$ (logic) $\implies x\in A\cup B$ (def. of union)\\
Since every element of A is an element of $A\cup B$, it follows by the definition of subset that $A\subset (A\cup B)$\\

%\noindent\textbf{Exercise \ref{exercise:Sets:}:}\\ %KW unnamed exercise

\noindent\textbf{Exercise \ref{exercise:Sets:26}:}\\
(Part 6 of Prop. 24)\\
a. Suppose $x$ is an element of $A\cup B$\\
$\implies x\in A\cup B$ (supposition)\\
$\implies x\in A$ or $x\in B$ (def. of union)\\
$\implies x\in B$ or $x\in A$ (logic)\\
$\implies x\in B\cup A$ (def. of union)\\
Since every element of $A\cup B$ is an element of $B\cup A$, we have $(A\cup B)\subset(B\cup A)$ (1)\\
Similarly, we can prove every element of $B\cup A$ is an element of $A\cup B$, so we have $(B\cup A)\subset (A\cup B)$ (2)\\
From (1) and (2) $\implies A\cup B = B\cup A$\\
b. Similarly, steps by steps as in part (a), we can prove that $A\cap B=B\cap A$\\

\noindent\textbf{Exercise \ref{exercise:Sets:s27}:}\\ %KW already done did not change to Adam's (aaftmt_suppl shows as 4.2.7 not 4.2.6)
%Prove Proposition~\ref{proposition:Sets:sets_de_morgan} part (2).
Assume $x$ is in $(A\cup B)'$\\
Then $x \notin A \cup B)$ [def. complement]\\
Then $x \notin A$ or $B$ [def. union]\\
Then $x \notin A$ and $x \notin B$ [logic]\\
Then $x \in A'$ and $x \in B'$ [def. complement]\\
Then $x \in A' \cap B'$ [def. interection]\\

\noindent\textbf{Exercise \ref{exercise:Sets:30}:}%KW updated d,e,g
%Prove the following statements by mimicking the style of proof in Example~\ref{example:Sets:other_relations}; that is use the definitions of $\cap, \cup, \setminus$, and $'$ as well as their properties listed in Proposition~\ref{proposition:Sets:sets_theorem_set_ops} and  Proposition~\ref{proposition:Sets:sets_de_morgan}. This type of proof is called an ``algebraic'' proof.  Every time you use a property, remember to give a reference!

%(You may find it easiest to begin with the more complicated side of the equality, and  simplify until it agrees with the other side. if you make that work, then start with the other side and simplify until the simplified versions of both sides finally agree.)

\begin{enumerate}[(a)]
\item
%$(A \cap B) \setminus B = \emptyset$.
\begin{align*}
(A\cap B) \setminus B & =(A\cap B)\cap B'\\
& = A\cap (B\cap B')\\
& = A\cap \emptyset\\
& = \emptyset
\end{align*}

\item
%$(A \cup B) \setminus B = A \setminus B$.
\begin{align*}
(A\cup B) \setminus B & =(A\cup B)\cap B'\\
& = B'\cap (A\cup B)\\
& = (B'\cap A)\cup(B'\cap B)\\
& = (B'\cap A)\cup \emptyset\\
& = B'\cap A\\
& = A\cap B'\\
& = A\setminus B
\end{align*}

\item
%$A \setminus (B \cup C) = (A \setminus B) \cap (A \setminus C)$. 
\begin{align*}
A\setminus (B\cup C) & =A\cap (B\cup C)'\\
& = A\cap (B'\cap C')\\
& = A\cap A\cap(B'\cap C')\\
& = (A\cap B')\cap (A\cap C')\\
& = (A\setminus B)\cap (A\setminus C)
\end{align*}
\item
%$(A \cap B) \setminus (B \cap C) = (A \cap B) \setminus (B \cap C)$.

\item
% $A \cap (B \setminus C) = (A \cap B) \setminus (A \cap C)$. 
\begin{align*}
A \cap (B \setminus C) &= \\
&= A \cap (B \cap C')\\
&= (A \cap B \cap C')\\
&= \emptyset \cup (A \cap B \cap C')\\
&= (\emptyset \cap B) \cup (A \cap B \cap C')\\
&= (A \cap A' \cap B) \cup (A \cap B \cap C')\\
&= (A \cap B \cap A') \cup (A \cap B \cap C')\\
&= [(A \cap B) \cap A'] \cup [(A \cap B) \cap C']\\
&= (A \cap B) \cap (A' \cup C')\\
&= (A \cap B) \cap (A \cap C)'\\
&= (A \cap B) \setminus (A \cap C)
\end{align*}

\item
%$(A \setminus B) \cup (B \setminus A) = (A \cup B) \setminus (A \cap B)$. 
\begin{align*}
(A \setminus B) \cup (B \setminus A) &= (A\cap B') \cup (B \cap A')\\
&= [(A\cap B') \cup B] \cap [(A \cap B') \cup A']\\
&= [B \cup (A\cap B')] \cap [A' \cup (A \cap B')]\\
&= [(B \cup A) \cap (B \cup B')] \cap [(A' \cup A) \cap (A' \cup B')]\\
&= [(B \cup A) \cap U] \cap [U \cap (A' \cup B')]\\
&= [(B \cup A) \cap U] \cap [(A' \cup B') \cap U]\\
&= (B \cup A) \cap (A' \cup B') \\
&= (A \cup B) \cap (A' \cup B') \\
&= (A \cup B) \cap (A \cap B)' \\
&= (A \cup B) \setminus (A \cap B)
\end{align*}

\item
%$(A \cup B \cup C) \cap D) = (A \cap D) \cup (B \cap D)\cup (C \cap D)$. 

\item
%$(A \cap B \cap C) \cup D = (A \cup D) \cap (B \cup D)\cap (C \cup D)$. 
\begin{align*}
(A \cap B \cap C) \cup D &= D \cup (A \cap B \cap C) \\
&= D \cup (A \cap (B \cap C)) \\
&= (D \cup A) \cap (D \cup (B \cap C)) \\
&= (D \cup A) \cap ((D \cup B) \cap (D \cup C)) \\
&= (D \cup A) \cap (D \cup B) \cap (D \cup C) \\\
&= (A \cup D) \cap (B \cup D) \cap (C \cup D)
\end{align*}
\end{enumerate}

\noindent\textbf{Exercise \ref{exercise:Sets:31}:} %KW updated d
\begin{enumerate}[(a)]
\item
%List the \emph{subsets} of $S =  \{a,b,c\}$. Include the empty set and non-proper subsets of $S$. How many subsets are in your list?
Subsets of S: $\emptyset; \{a\};\{b\};\{c\};\{a,b\};\{a,c\};\{b,c\};\{a,b,c\};$\\
There are 8 subsets.

\item
%If you listed the subsets of $\{a,b\}$, how many subsets would be in your list?
There are 4 subsets.

\item
%If you listed the subsets of $\{a,b,c,d\}$, how many subsets would be in your list?
There are 16 subsets.

\item
%**If you listed the subsets of $\{a,b,c,\ldots,x,y,z\}$, how many subsets would be in your list?
%\hyperref[sec:Sets:Hints]{(*Hint*)}
Based on the patern of $(a), (b)$, and $(c)$, there would be $2^{26}=67,108,864$ subsets.
\end{enumerate}

\noindent\textbf{Exercise \ref{exercise:Sets:cup_group}:} %KW updated
%Let $G$ be the set of subsets of the set $\{a,b,c\}$.
\begin{enumerate}[(a)]
\item
%Does the set $G$  with the operation $\cup$ have the closure property? \emph{Justify} your answer.
Yes, because if you union any subset in $G$, which are listed in Exercise \ref{exercise:Sets:31}(a), with another subset in $G$, then you will get a subset of G. For example, $\{1,2\}\cup \{2,3\}=\{1,2,3\}$

\item
%Does the set $G$  with the operation $\cup$ have an identity? If so, what is it? Which part of  Proposition~\ref{proposition:Sets:sets_theorem_set_ops} enabled you to draw this conclusion?
Yes. The identity element to the set $G$ is the $\emptyset$. Proposition~\ref{proposition:Sets:sets_theorem_set_ops} [part $3$] allows us to draw this conclusion.

\item
%Is the operation $\cup$ defined on the set $G$ associative? Which part of  Proposition~\ref{proposition:Sets:sets_theorem_set_ops} enabled you to draw this conclusion?
Yes, Proposition~\ref{proposition:Sets:sets_theorem_set_ops} [part $5$] allows us to make this conclusion.

\item
%Is the operation $\cup$ defined on the set $G$ commutative? Which part of  Proposition~\ref{proposition:Sets:sets_theorem_set_ops} enabled you to draw this conclusion?
Yes, Proposition~\ref{proposition:Sets:sets_theorem_set_ops} [part $6$] allows us to draw the this conclusion.

\item
%Does each element of $G$ have a unique inverse under the operation $\cup$? If so, which part of  Proposition~\ref{proposition:Sets:sets_theorem_set_ops} enabled you to draw this conclusion? If not, provide a counterexample.
Definition of inverse elements from \textit{Modern Algebra an Introduction 5th ed.} by John R, Durbin, For each $a\in G$ there is an element $b\in G$ such that $a \cdot b = b \cdot a = e$ where $e$ is the identity element.\\
        
To answer the question, No. The subset $\{1,2,3\}\cup \{2,3\}=\{2,3\}\cup \{1,2,3\}=\{1,2,3\}\neq {\emptyset}$ Since $a\cup b = b\cup a\neq e$ where is the identity. The operation $\cup$ does not have an inverse for each element of $G$.

\item
%Is the set $G$ a group under the $\cup$ operation?  \emph{Justify} your answer.
No, because it does not have an inverse for every element in the group $G$.
\end{enumerate}

\noindent\textbf{Exercise \ref{exercise:Sets:cap_group}:}\\

\noindent\textbf{Exercise \ref{exercise:Sets:symm_diff}:} %KW updated
%Besides $\cup$ and $\cap$, there is another set operation called \emph{symmetric difference},\index{Set operations!symmetric difference} which is sometimes denoted by the symbol $\Delta$ and is defined as:
%\begin{equation*}
%A \Delta B = (A \setminus B) \cup (B\setminus A).
%\end{equation*}
%Given a set $U$, let $G$ be the set of all subsets of $U$.  Repeat parts (a)--(f) of Exercise~\ref{exercise:Sets:cap_group}, but this time for the set operation $\Delta$ instead of $\cap$.
\begin{enumerate}[(a)]
\item
%Does the set $G$ with the operation $\Delta$ have the closure property? \textit{Justify} your\\ answer.
Yes. Because $A\setminus A\subset G$ for all subsets $A$.

\item
%Does the set $G$ with the operation $\Delta$ have an identity? If so, what is it? Which part of  Proposition~\ref{proposition:Sets:sets_theorem_set_ops} enabled you to draw this conclusion?
Yes. $A\Delta \emptyset=A$ and $A\Delta A=\emptyset\ \forall\ $ subsets $A$, therefore the $\emptyset$ is the identity element. Proposition~\ref{proposition:Sets:sets_theorem_set_ops} [parts $2$ and $3$] allows us to draw this conclusion.

\item
%Is the operation $\Delta$ defined on the set $G$ associative? Which part of  Proposition~\ref{proposition:Sets:sets_theorem_set_ops} enabled you to draw this conclusion?
Yes. $A\Delta (B \Delta C) = G = (A\Delta B)\Delta C$, for all subsets $G$. Proposition~\ref{proposition:Sets:sets_theorem_set_ops} [part $5$] allows us to make this conclusion.

\item
%Is the operation  $\Delta$ defined on the set $G$ commutative? Which part of  Proposition~\ref{proposition:Sets:sets_theorem_set_ops} enabled you to draw this conclusion?
Yes, $A\Delta A=\emptyset=A\Delta A$ for all subsets $A$. Proposition~\ref{proposition:Sets:sets_theorem_set_ops} [part $6$] allows us to draw the this conclusion.

\item
%Does each element of $G$ have a unique inverse under the operation $\Delta$? If so, which part of  Proposition~\ref{proposition:Sets:sets_theorem_set_ops} enabled you to draw this conclusion? If not, provide a counterexample.
Yes. For all subsets $A$, $A\Delta A=\emptyset$ which is the identity element.

\item
%Is the set $G$ a group under the  $\Delta$ operation?  \emph{Justify} your answer.
%No, because it does not have an inverse for every element in the group $G$.
Yes $G$ is a group under the operation $\Delta$, because it is closed, has an identity, is associative, is communative, and has a unique inverse for all subsets $A\in G$
\end{enumerate}