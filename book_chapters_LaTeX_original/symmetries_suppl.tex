\section{Solutions for  ``Symmetries of Plane Figures''}
\noindent\textbf{\textit{ (Chapter \ref{symmetries})}}\bigskip

\noindent\textbf{Exercise \ref{exercise:symmetries:mercedes}:}
%List six different symmetries of the Mercedes logo.
\begin{multicols}{3}
\begin{enumerate}
\item
$\var{id}$

\item
$r_{120}$

\item
$r_{240}$

\item
$s_v$

\item
$s_{120^{\circ}}$

\item
$s_{240^{\circ}}$
\end{enumerate}
\end{multicols}

\noindent\textbf{Exercise \ref{exercise:symmetries:hexagon}:}
\begin{enumerate}[{a.}]
\item
%Is $r_{60}$ one-to-one?  Explain why or why not.
Yes, $r_{60}$ is one-to-one, because each element of the domain goes to only one element of the codomain.

\item
%Is $r_{60}$ onto?  Explain why or why not.
Yes, $r_{60}$ is onto, because every element of the codomain maps to an element in the domain.

\item
%Is $r_{60}$ a bijection?  Explain why or why not.
Yes, $r_{60}$ is it is one-to-one and onto it is a bijection.
\end{enumerate}

\noindent\textbf{Exercise \ref{exercise:symmetries:bijectnotsym}:}\\
%Create a bijection from $\{A,B,C,D,E,F\} \to \{A,B,C,D,E,F\}$ that does not correspond to a symmetry of the regular hexagon in  Figure~\ref{hex60rot}. \emph{Explain} why it is not a symmetry.
It's not a symmetry because it doesn't preserve the length and position of corresponding side $\overline{AB}$.\\
\\
$\begin{pmatrix}
A & B & C & D & E & F\\
A & C & B & D & E & F
\end{pmatrix}$\\

\noindent\textbf{Exercise \ref{exercise:symmetries:10}:}
\begin{enumerate}[{a.}]
\item
%Explain why a $90^\circ$ rotation, a $270^\circ$ rotation, or reflection across a diagonal are not symmetries  of the rectangle $ABCD$.
The two rotations and the reflection all change the final image so they are not symmetries.

\item
%What subcategory of rectangle would have a $90^\circ$ rotation, $270^\circ$ rotation, and a reflection across a diagonal as symmetries?
square

\item
%What rotation angle does the identity symmetry correspond to? (Give the easiest answer.)
$r_{360}$ or $\var{id}$

\item
%Write each of the symmetries of a rectangle as a function (use either a table, ordered pairs, arrow diagram, etc.)
$\var{id}: \begin{pmatrix}
A & B & C & D\\
A & B & C & D
\end{pmatrix}$
$\qquad r_{180}: \begin{pmatrix}
A & B & C & D\\
C & D & A & B
\end{pmatrix}$\\
\\
$s_{v}: \begin{pmatrix}
A & B & C & D\\
D & C & B & A
\end{pmatrix}$
$\qquad s_{h}: \begin{pmatrix}
A & B & C & D\\
B & A & D & C
\end{pmatrix}$\\
\end{enumerate}

\noindent\textbf{Exercise \ref{exercise:symmetries:SymmComposition}:}
%With reference to the symmetries of a rectangle in Example~\ref{example:symmetries:rectsymmetries}, let  $r_{180}$ be the $180^\circ$ counterclockwise rotation and let $s_v$ be the reflection across the vertical axis.
%(Note that reflection across the vertical axis is sometimes called ``horizontal reflection,'' since the figure ``flips'' from left to right. Admittedly this is confusing, but that's what people call it so what can you do?)
\begin{enumerate}[{a.}]
\item
%Write the function $r_{180}$ in ordered pair notation.
$r_{180} = \{(A, C), (B, D), (C, A), (D, B)\}$

\item
%Write the function $s_v$ in ordered pair notation.
$s_v = \{(A, D), (B, C), (C, B), (D, A)\}$

\item 
%Write the function $r_{180} \compose s_v$ in ordered pair notation. Is it a symmetry of the rectangle? If so, then which one?
$r_{180}: \begin{pmatrix}
A & B & C & D\\
C & D & A & B
\end{pmatrix}$
$\qquad s_v: \begin{pmatrix}
A & B & C & D\\
D & C & B & A
\end{pmatrix}$\\
$r_{180}\circ s_v = \{(A,B), (B,A), (C,D), (D,C)\}$\\
It's the symmetry $s_h$.

\item
%Write the function $s_v \compose r_{180}$ in ordered pair notation. Is it a symmetry of the rectangle? If so, then which one?
$s_v\circ r_{180}= \{(A,B), (B,A), (C,D), (D,C)\}$\\
It's the symmetry $s_h$.\\
\end{enumerate}

\noindent\textbf{Exercise \ref{exercise:symmetries:CompSymm}:}
\begin{enumerate}[{a.}]
\item
%Write $s_h$ in tableau form, where $s_h$ is reflection across the horizontal axis. (Note $s_h$ is sometimes referred to  as ``vertical reflection,'', since the two reflected halves are stacked on top of each other.)
$s_h: \begin{pmatrix}
A & B & C & D\\
B & A & D & C
\end{pmatrix}$

\item
%Does $r_{180} \compose s_v = s_h$?
Yes, $r_{180}\circ s_v=s_h$

\item
%Compute $s_h \compose  s_v$. Is this a symmetry? If so, which one?
$s_h\circ s_v=\begin{pmatrix}
A & B & C & D\\
C & D & A & B
\end{pmatrix}$\\
It's the symmetry $r_{180}$.

\item
%Compute $s_v \compose r_{180}$.  Is this a symmetry? If so, which one?
$s_v\circ r_{180}=\begin{pmatrix}
A & B & C & D\\
B & A & D & C
\end{pmatrix}$\\
It's the symmetry $s_h$.
\end{enumerate}

\noindent\textbf{Exercise \ref{exercise:symmetries:16}:}
%With reference to the hexagon in Figure~\ref{hex60rot}, for the symmetries $f$ and $g$ in parts (a)-(d) below:
%\begin{enumerate}[(i)]
%\item
%Write the symmetries $f$ and $g$ in tableau form.
%\item
%Compute $f \compose g$ and $g \compose f$, expressing your answers in tableau form.
%\item 
%Describe the symmetries that correspond to $f \compose g$ and $g \compose f$, respectively.
%%\end{enumerate}
%\medskip
%\noindent
%Note ${\var id}$ denotes the identity symmetry, that is the symmetry that leaves all points unchanged.
%\medskip

\begin{enumerate}[{a.}]
\item
%$f=$ rotation by $240^\circ, g=$ rotation by $120^\circ$
	\begin{enumerate}[(i)]
	\item
	$f=r_{240}=\begin{pmatrix}
	A & B & C & D & E & F\\
	E & F & A & B & C & D
	\end{pmatrix}$
	$\qquad g=r_{120}=\begin{pmatrix}
	A & B & C & D & E & F\\
	C & D & E & F & A & B
	\end{pmatrix}$\\
	
	\item
	$f\circ g=\begin{pmatrix}
	A & B & C & D & E & F\\
	A & B & C & D & E & F
	\end{pmatrix}$
	$\qquad g\circ f=\begin{pmatrix}
	A & B & C & D & E & F\\
	A & B & C & D & E & F
	\end{pmatrix}$
	
	\item
	$f\circ g = \var{id}$\\
	$g\circ f= \var{id}$
	\end{enumerate}
	
\item
%$f={\var id}, g=$ rotation by $120^\circ$
	\begin{enumerate}[(i)]
	\item
	$f=\var{id}=\begin{pmatrix}
	A & B & C & D & E & F\\
	A & B & C & D & E & F
	\end{pmatrix}$
	$\qquad g=r_{120}=\begin{pmatrix}
	A & B & C & D & E & F\\
	C & D & E & F & A & B
	\end{pmatrix}$\\
	
	\item
	$f\circ g=\begin{pmatrix}
	A & B & C & D & E & F\\
	C & D & E & F & A & B
	\end{pmatrix}$
	$\qquad g\circ f=\begin{pmatrix}
	A & B & C & D & E & F\\
	C & D & E & F & A & B
	\end{pmatrix}$
	
	\item
	$f\circ g = r_{120}$\\
	$g\circ f = r_{120}$
	\end{enumerate}
	
\item
%$f=$rotation by $ 240^\circ, g=$reflection across the line $BE$
	\begin{enumerate}[(i)]
	\item
	$f=r_{240}=\begin{pmatrix}
	A & B & C & D & E & F\\
	E & F & A & B & C & D
	\end{pmatrix}$
	$\qquad g=s_{BE}=\begin{pmatrix}
	A & B & C & D & E & F\\
	C & B & A & F & E & D
	\end{pmatrix}$\\
	
	\item
	$f\circ g=\begin{pmatrix}
	A & B & C & D & E & F\\
	A & F & E & D & C & B
	\end{pmatrix}$
	$\qquad g\circ f=\begin{pmatrix}
	A & B & C & D & E & F\\
	E & D & C & B & A & F
	\end{pmatrix}$
	
	\item
	$f\circ g = s_{AD}$\\
	$g\circ f = s_{CF}$
	\end{enumerate}
	
\item
%$f=$rotation by $ 180^\circ, g=$reflection across the line $CF$
	\begin{enumerate}[(i)]
	\item
	$f=r_{180}=\begin{pmatrix}
	A & B & C & D & E & F\\
	D & E & F & A & B & C
	\end{pmatrix}$
	$\qquad g=s_{CF}=\begin{pmatrix}
	A & B & C & D & E & F\\
	E & D & C & B & A & F
	\end{pmatrix}$\\
	
	\item
	$f\circ g=\begin{pmatrix}
	A & B & C & D & E & F\\
	B & A & F & E & D & C
	\end{pmatrix}$
	$\qquad g\circ f=\begin{pmatrix}
	A & B & C & D & E & F\\
	B & A & F & E & D & C
	\end{pmatrix}$
	
	\item
	$f\circ g = s_{30}$\\
	$g\circ f = s_{30}$
	\end{enumerate}
\end{enumerate}

\noindent\textbf{Exercise \ref{exercise:symmetries:S3Table}:}
%Verify the following entries in Table~\ref{S3_table} by (i) writing the symmetries in tableau form and (ii) computing the composition directly.
\begin{enumerate}[{a.}]
\item
%row 2, column 4\\
$\rho_1 \circ \mu_1 = \begin{pmatrix}
A & B & C\\
B & C & A
\end{pmatrix} \circ \begin{pmatrix}
A & B & C\\
A & C & B
\end{pmatrix}=\begin{pmatrix}
A & B & C\\
B & A & C
\end{pmatrix}=\mu_3$

\item
%row 4, column 2\\
$\mu_1\circ \rho_1=\begin{pmatrix}
A & B & C\\
A & C & B
\end{pmatrix}\circ \begin{pmatrix}
A & B & C\\
B & C & A
\end{pmatrix}=\begin{pmatrix}
A & B & C\\
C & B & A
\end{pmatrix}=\mu_2$

\item
%row 3, column 6\\
$\rho_2\circ \mu_3=\begin{pmatrix}
A & B & C\\
C & A & B
\end{pmatrix}\circ \begin{pmatrix}
A & B & C\\
B & A & C
\end{pmatrix}=\begin{pmatrix}
A & B & C\\
A & C & B
\end{pmatrix}=\mu_1$
\end{enumerate}

\noindent\textbf{Exercise \ref{exercise:symmetries:20}:}
%Use Table~\ref{S3_table} to answer the following questions. 
\begin{enumerate}[{a.}]
\item
%Explain why Table~\ref{S3_table} shows that ${\var id}$ satisfies the definition of an identity element.
Any symmetry composes with $\var{id}$ leave that symmetry unchanged.\\
$\var{id}$ composes with any symmetry still leave that symmetry unchanged.

\item
%Does every element in $S$ have an inverse?  List the inverses for each symmetry that has an inverse.
Yes, because every element has another element that when they are composed returns the $\var{id}$. The inverses are as follows:\\
	\begin{multicols}{3}
	\begin{itemize}
	\item
	$\var{id} \leftrightarrow \var{id}$.
	
	\item
	$\rho_1 \leftrightarrow \rho_2$.
	
	\item
	$\mu_1 \leftrightarrow \mu_1$.
	
	\item
	$\mu_2 \leftrightarrow \mu_2$.
	
	\item
	$\mu_3 \leftrightarrow \mu_3$.
	\end{itemize}
	\end{multicols}
	
\item
%Explain why Table~\ref{S3_table} shows that composition is \emph{not} commutative.
It is not commutative, because you cannot always switch the order of elements, an example: $\mu_1\circ \mu_2 = \rho_1$ but $\mu_2\circ \mu_1=\rho_2$
\end{enumerate}

\noindent\textbf{Exercise \ref{exercise:symmetries:prop_proof}:}\\
%Write out the proof of Proposition \ref{proposition:symmetries:symcompinv} part (i).
%\hyperref[sec:symmetries:hints]{(*Hint*)}
We show that $s^{-1}$ leave distances between points unchanged as follows:\\
Proof:\\
Choose any two points $A$ and $B$ in the figure, and let $A'=s^{-1}(A)$ and $B'=s^{-1}(B)$.\\
By definition of inverse, it follows that $s(A')=A$ and $s(B')=B$.\\
Since $s$ is a symmetry, it follows that $\overline{A'B'}$ = $\overline{AB}$\\
Since $A$ and $B$ were arbitrary points in the figure, we have shown that $s^{-1}$ leaves distance between points unchanged.

\noindent\textbf{Exercise \ref{exercise:symmetries:symm_square}:}
\begin{enumerate}[{a.}]
\item
%Describe all symmetries of a square  (for example: ``reflection about the vertical axis '' describes one symmetry: give similar descriptions of all symmetries of the square)
$\var{id}, r_{90}, r_{180}, r_{270}, s_v, s_h, s_{45}, s_{135}$

\item
%Label the square's vertices as $A, B, C, D$, and write down each symmetry in tableau form. As in Figure~\ref{groups_s3_symmetry_fig}, denote each symmetry by a variable (you may use $\rho_1, \rho_2, \ldots$ for the rotations and $\mu_1, \mu_2, \ldots$ for the reflections).
	\begin{multicols}{2}
	\begin{itemize}
	\item
	$\var{id}=\begin{pmatrix}
	A & B & C & D\\
	A & B & C & D
	\end{pmatrix}$
	
	\item
	$r_{90}=\begin{pmatrix}
	A & B & C & D\\
	B & C & D & A
	\end{pmatrix}$
	
	\item
	$r_{180}=\begin{pmatrix}
	A & B & C & D\\
	C & D & A & B
	\end{pmatrix}$
	
	\item
	$r_{270}=\begin{pmatrix}
	A & B & C & D\\
	D & A & B & C
	\end{pmatrix}$
	
	\item
	$s_{v}=\begin{pmatrix}
	A & B & C & D\\
	D & C & B & A
	\end{pmatrix}$
	
	\item
	$s_{h}=\begin{pmatrix}
	A & B & C & D\\
	B & A & D & C
	\end{pmatrix}$
	
	\item
	$s_{45}=\begin{pmatrix}
	A & B & C & D\\
	C & B & A & D
	\end{pmatrix}$
	
	\item
	$s_{135}=\begin{pmatrix}
	A & B & C & D\\
	A & D & C & B
	\end{pmatrix}$
	\end{itemize}
	\end{multicols}
	
\item
%Write the Cayley table for the symmetries of a square.
	\begin{tabular}{c| c c c c c c c c }
		o & $\var{id}$ & $r_{90}$ & $r_{180}$ & $r_{270}$ & $s_v$ & $s_h$ & $s_{45}$ & $s_{135}$\\
		\hline
		$\var{id}$ & $\var{id}$ & $r_{90}$ & $r_{180}$ & $r_{270}$ & $s_v$ & $s_h$ & $s_{45}$ & $s_{135}$\\
		$r_{90}$ & $r_{90}$ & $r_{180}$ & $r_{270}$ & $\var{id}$ & $s_{135}$ & $s_{45}$ & $s_v$ & $s_h$\\
		$r_{180}$ & $r_{180}$ & $r_{270}$ & $\var{id}$ & $s_{90}$ & $s_h$ & $s_v$ & $s_{135}$ & $s_{45}$\\
		$r_{270}$ & $r_{270}$ & $\var{id}$ & $r_{90}$ & $r_{180}$ & $s_{45}$ & $s_{135}$ & $s_h$ & $s_v$\\
		$s_v$ & $s_v$ & $s_{45}$ & $s_h$ & $s_{135}$ & $\var{id}$ & $r_{180}$ & $r_{90}$ & $r_{270}$\\
		$s_h$ & $s_h$ & $s_{135}$ & $s_v$ & $s_{45}$ & $r_{180}$ & $\var{id}$ & $r_{270}$ & $r_{90}$\\
		$s_{45}$ & $s_{45}$ & $s_h$ & $s_{135}$ & $s_v$ & $r_{270}$ & $r_{90}$ & $\var{id}$ & $r_{180}$\\
		$s_{135}$ & $s_{135}$ & $s_v$ & $s_{45}$ & $s_h$ & $r_{90}$ & $r_{270}$ & $r_{180}$ & $\var{id}$
	\end{tabular}

\item
%For each symmetry of a square, list its inverse.
	\begin{multicols}{3}
	\begin{itemize}
	\item
	$\var{id} \leftrightarrow \var{id}$.
	
	\item
	$r_{90} \leftrightarrow r_{270}$.
	
	\item
	$r_{180} \leftrightarrow r_{180}$.
	
	\item
	$s_v \leftrightarrow s_v$.
	
	\item
	$s_h \leftrightarrow s_h$.
	
	\item
	$s_{45} \leftrightarrow s_{45}$.
	
	\item
	$s_{135} \leftrightarrow s_{135}$.
	\end{itemize}
	\end{multicols}
\end{enumerate}

\noindent\textbf{Exercise \ref{exercise:symmetries:describesymm}:}
%With reference to the logos in Figure~\ref{logos}:
\begin{multicols}{3}
\begin{enumerate}[{a.}]
\item
%For which logos do the set of symmetries include all symmetries of the equilateral triangle? (Note: there are at least two!)
$i, v$

\item
%For which logos do the set of symmetries include all symmetries of the rectangle?
$v$

\item
%For which logos do the set of symmetries include all symmetries of the hexagon?
$v$

\item
%Which logos have set of symmetries which are proper subsets of the set of all symmetries as the rectangle?
$ii, vi$

\item
%Give two logos such that all symmetries of the first logo are also symmetries of the second logo.
first $i$, second $v$

\item
%Which logos have no symmetries except for the identity?
$vi$
\end{enumerate}
\end{multicols}

\noindent\textbf{Exercise \ref{exercise:symmetries:reflection_pentagon}:}
\begin{enumerate}[{a.}]
\item
%Write the reflection $s$  for the pentagon in tableau form.  
$s =\begin{pmatrix}
	1 & 2 & 3 & 4 & 5\\
	1 & 5 & 4 & 3 & 2
	\end{pmatrix}$
	
\item
%How many vertices are fixed by $s$? What are they?
1 vertice is fixed and it is 1.

\item
%What is $s^2$?  (Recall that $s^2$ means the same as $s \compose s$.)
$s^2 = s \circ s =\begin{pmatrix}
	1 & 2 & 3 & 4 & 5\\
	1 & 2 & 3 & 4 & 5
	\end{pmatrix} = \var{id}$
\end{enumerate}

\noindent\textbf{Exercise \ref{exercise:symmetries:31}:}
\begin{enumerate}[{a.}]
\item
%Write the reflection $s$ for the octagon in tableau form. 
$s =\begin{pmatrix}
	1 & 2 & 3 & 4 & 5 & 6 & 7 & 8\\
	1 & 8 & 7 & 6 & 5 & 4 & 3 & 2
	\end{pmatrix}$
	
\item
%How many vertices are fixed by $s$? What are they?
Two vertices are fixed: 1 and 5.

\item
%What is $s^2$?
$s^2 = s\circ s = \begin{pmatrix}
	1 & 2 & 3 & 4 & 5 & 6 & 7 & 8\\
	1 & 2 & 3 & 4 & 5 & 6 & 7 & 8
	\end{pmatrix} = \var{id}$
\end{enumerate}

\noindent\textbf{Exercise \ref{exercise:symmetries:32}:}
%Prove the following proposition by filling in the blanks:
%\medskip
%
%\noindent \textbf{Proposition.} 
%If $0< p,q < n$ and $p \neq q$, then $s \compose r^p$ and $s \compose r^q$ are distinct elements of $D_n$: that is, $s\circ r^p \neq s \compose r^q$.
%\medskip
%
%\begin{proof}
%\begin{itemize}
%\item
% The proof is by contradiction. Given  $0< p,q < n$ and  $p \neq q$, and suppose that $s \compose r^p  \underline{~<1>~} s \compose r^q$
%\item
%Compose both sides of the equation with $s$, and obtain the equation:  $s \compose (s \compose r^p) =  \underline{~<2>~} $.
%\item
%By the associative property of composition, this can be rewritten: $(s \compose s) \compose \underline{~<3>~}  =  \underline{~<4>~}$
%\item
%Since $s \compose s = \underline{~<5>~}$, this can be rewritten: ${\var id} \compose \underline{~<6>~}  =  \underline{~<7>~}$.
%\item
%Since ${\var id}$ is a group identity, we have: $r^p = \underline{~<8>~}$.
%\item
%But we have already shown that $r^p$ and $r^q$ are distinct symmetries if $0< p,q < n$ and  $p \neq q$. This is a contradiction.
%\item
%Therefore we conclude that our supposition was incorrect, and $s \compose r^p  \underline{~<9>~} s \compose r^q$. This completes the proof.
%\end{itemize}
\begin{multicols}{3}
\begin{enumerate}
\item
$=$

\item
$s \circ (s \circ r^q)$

\item
$r^p$

\item
$(s \circ s) \circ r^q$

\item
$\var{id}$

\item
$r^p$

\item
$\var{id} \circ r^q$

\item
$r^q$

\item
$\neq$
\end{enumerate}
\end{multicols}

\noindent\textbf{Exercise \ref{exercise:symmetries:33}:}\\
%Prove the following proposition: 
%\medskip
%
%\noindent
%\textbf{Proposition}
%If $0< q < n$  then $s$ and $s \compose r^q$ are distinct elements of $D_n$: that is, $s \neq s \compose r^q$.
%\medskip
%\hyperref[sec:symmetries:hints]{(*Hint*)}
Suppose $0<q<n$ and $s=s\circ r^q$, we have:
$s\circ s=s\circ(s\circ r^q)$ (compose s to the left of both sides)\\
$\implies s\circ s=(s\circ s)\circ r^q$ (associative property)\\
$\implies id=id\circ r^q$ (because $s\circ s=id$)\\
$\implies id=r^q$ ($id$ is group identity)\\
It's a contradiction since we have shown that $id\neq r^q$ when $0<q<n$.\\
Therefore, $s\neq s\circ r^q$.\\
\\

\noindent\textbf{Exercise \ref{exercise:symmetries:34}:}\\
%Fill in the blanks to prove that given any integers $p,q$ with $0< p,q < n$, $s \compose r^p \neq r^q$ :
%\begin{itemize}
%\item
%The proof is by contradiction: so given integers $p,q$ with $0< p,q < n$, we suppose $\underline{~<1>~}$.
%\item
%By multiplying both sides on the \emph{right} by $r^{n-p}$, we obtain $s \compose r^p \compose \underline{~<2>~} = r^q \compose \underline{~<3>~}$
%\item
%By associativity, we have $s \compose  \underline{~<4>~} =  \underline{~<5>~}$
%\item
%Using the fact that \underline{~<6>~} = ${\var id}$, we obtain $s = \underline{~<7>~}$
%\item
%The left side of this equation is a reflection, and the right side is a $\underline{~<8>~}$, which is a contradiction.
%\item
%This contradiction implies that our supposition is incorrect, so  given integers $p,q$ with $0< p,q < n$, we conclude $\underline{~<9>~}$.
%\end{itemize}
\begin{multicols}{3}
\begin{enumerate}
\item
$s\circ r^p=r^q$

\item
$r^{n-p}$

\item
$r^{n-p}$

\item
$(r^p\circ r^{n-p})$

\item
$(r^q\circ r^{n-p})$

\item
$r^p\circ r^{n-p}$

\item
$r^q\circ r^{n-p}$

\item
rotation

\item
$s\circ r^p\neq r^q$
\end{enumerate}
\end{multicols}

\noindent\textbf{Exercise \ref{exercise:symmetries:36}:}
\begin{enumerate}[{a.}]
\item
%List four reflections of the square in tableau form.
%\hyperref[sec:symmetries:hints]{(*Hint*)}
	\begin{multicols}{2}
	\begin{itemize}
	\item
	$s_h =\begin{pmatrix}
	1 & 2 & 3 & 4\\
	2 & 1 & 4 & 3
	\end{pmatrix}$
	
	\item
	$s_v =\begin{pmatrix}
	1 & 2 & 3 & 4\\
	4 & 3 & 2 & 1
	\end{pmatrix}$
	
	\item
	$s_{13} =\begin{pmatrix}
	1 & 2 & 3 & 4\\
	1 & 4 & 3 & 2
	\end{pmatrix}$
	
	\item
	$s_{24} =\begin{pmatrix}
	1 & 2 & 3 & 4\\
	3 & 2 & 1 & 4
	\end{pmatrix}$
	\end{itemize}
	\end{multicols}
	
\item
%Let $\mu$ be any of the reflections in part (a). What is $\mu \compose \mu$?
$\mu = s_h \implies \mu \circ \mu = \var{id}$

\item
%How many reflections have no fixed vertices?
2 reflections fix no vertices.

\item
%How many reflections fix exactly one vertex?
0 reflections fix 1 vertice.

\item
%How many reflections fix exactly two vertices?
2 relections fix 2 vertices.
\end{enumerate}

\noindent\textbf{Exercise \ref{exercise:symmetries:PentagonRefl}:}
\begin{enumerate}[{a.}]
\item
%List five reflections of the pentagon in tableau form.
%\hyperref[sec:symmetries:hints]{(*Hint*)}
	\begin{multicols}{2}
	\begin{itemize}
	\item
	$s_1 =\begin{pmatrix}
	1 & 2 & 3 & 4 & 5\\
	1 & 5 & 4 & 3 & 2
	\end{pmatrix}$
	
	\item
	$s_2 =\begin{pmatrix}
	1 & 2 & 3 & 4 & 5\\
	3 & 2 & 1 & 5 & 4
	\end{pmatrix}$
	
	\item
	$s_3 =\begin{pmatrix}
	1 & 2 & 3 & 4 & 5\\
	5 & 4 & 3 & 2 & 1
	\end{pmatrix}$
	
	\item
	$s_4 =\begin{pmatrix}
	1 & 2 & 3 & 4 & 5\\
	2 & 1 & 5 & 4 & 3
	\end{pmatrix}$
	
	\item
	$s_5 =\begin{pmatrix}
	1 & 2 & 3 & 4 & 5\\
	4 & 3 & 2 & 1 & 5
	\end{pmatrix}$
	\end{itemize}
	\end{multicols}
\item
%Let $\mu$ be any of the reflections in part (a). What is $\mu \compose \mu$?
$\mu = s_1 \implies \mu \circ \mu = \var{id}$

\item
%How many reflections have no fixed vertices?
0 reflections fix no vertex.

\item
%How many reflections fix exactly one vertex?
5 reflections fix 1 vertex.

\item
%How many reflections fix exactly two vertices?
0 reflections fix 2 vertices.
\end{enumerate}

\noindent\textbf{Exercise \ref{exercise:symmetries:HexagonRefl}:}
\begin{enumerate}[{a.}]
\item
%List six reflections of the hexagon in tableau form.
%\hyperref[sec:symmetries:hints]{(*Hint*)}
	\begin{multicols}{2}
	\begin{itemize}
	\item
	$s_1=\begin{pmatrix}
	1 & 2 & 3 & 4 & 5 & 6\\
	1 & 6 & 5 & 4 & 3 & 2
	\end{pmatrix}$

	\item
	$s_2=\begin{pmatrix}
	1 & 2 & 3 & 4 & 5 & 6\\
	3 & 2 & 1 & 6 & 5 & 4
	\end{pmatrix}$

	\item
	$s_3=\begin{pmatrix}
	1 & 2 & 3 & 4 & 5 & 6\\
	5 & 4 & 3 & 2 & 1 & 6
	\end{pmatrix}$
	
	\item
	$s_{12}=\begin{pmatrix}
	1 & 2 & 3 & 4 & 5 & 6\\
	2 & 1 & 6 & 5 & 4 & 3
	\end{pmatrix}$


	\item
	$s_{23}=\begin{pmatrix}
	1 & 2 & 3 & 4 & 5 & 6\\
	4 & 3 & 2 & 1 & 6 & 5
	\end{pmatrix}$
	
	\item
	$s_{34}=\begin{pmatrix}
	1 & 2 & 3 & 4 & 5 & 6\\
	6 & 5 & 4 & 3 & 2 & 1
	\end{pmatrix}$
	\end{itemize}
	\end{multicols}
	
\item
%Let $\mu$ be any of the reflections in part (a). What is $\mu \compose \mu$?
$\mu\circ\mu = \var{id}$

\item
%How many reflections have no fixed vertices?
3 reflections have no fixed vertices.
\end{enumerate}

\noindent\textbf{Exercise \ref{exercise:symmetries:nonagon}:}
\begin{enumerate}[{a.}]
\item 
%Complete the second row of the following tableau that represents the reflection of the nonagon that fixes vertex 4:
%
%$$\mu_1 = \begin{pmatrix} 1 & 2 & 3 & 4 & 5 & 6 & 7 & 8 & 9 \\ \_\_ & \_\_ & \_\_ & 4 & \_\_ & \_\_ & \_\_ & \_\_ & \_\_   \end{pmatrix}$$
$\mu_1 = \begin{pmatrix} 1 & 2 & 3 & 4 & 5 & 6 & 7 & 8 & 9 \\ 7 & 6 & 5 & 4 & 3 & 2 & 1 & 9 & 8   \end{pmatrix}$

\item
%Complete the second row of the following tableau that represents the reflection of the 10-gon that fixes vertex 4:
%
%$$\mu_2 = \begin{pmatrix} 1 & 2 & 3 & 4 & 5 & 6 & 7 & 8 & 9 & 10  \\ \_\_ & \_\_ & \_\_ & 4 & \_\_ & \_\_ & \_\_ & \_\_ & \_\_ & \_\_   \end{pmatrix}$$
$\mu_2 = \begin{pmatrix} 1 & 2 & 3 & 4 & 5 & 6 & 7 & 8 & 9 & 10  \\ 7 & 6 & 5 & 4 & 3 & 2 & 1 & 10 & 9 & 8 \end{pmatrix}$

\item 
%Complete the second row of the following tableau that represents the reflection of the 10-gon that exchanges vertices 6 and 7:
%
%$$\mu_3 = \begin{pmatrix} 1 & 2 & 3 & 4 & 5 & 6 & 7 & 8 & 9 & 10 \\ \_\_ & \_\_ & \_\_  & \_\_ & \_\_ & 7 & 6 & \_\_ & \_\_ & \_\_  \end{pmatrix}$$
$\mu_3 = \begin{pmatrix} 1 & 2 & 3 & 4 & 5 & 6 & 7 & 8 & 9 & 10 \\ 2 & 1 & 10 & 9 & 8 & 7 & 6 & 5 & 4 & 3\end{pmatrix}$

\item
%What is $\mu_1 \compose \mu_1$? What is $\mu_2 \compose \mu_2$? What is $\mu_3 \compose \mu_3$?
$\mu_1\circ\mu_1 = \var{id}$\\
$\mu_2\circ\mu_2 = \var{id}$\\
$\mu_3\circ\mu_3 = \var{id}$
\end{enumerate}


\noindent\textbf{Exercise \ref{exercise:symmetries:41}:}
\begin{enumerate}[{a.}]
\item
%Based on results we've shown, prove that $s \compose r^p$ must be a reflection, for $0 < p < n$.
$s\circ r^p$ is either a rotation or a reflection. (Proposition 42)\\
Suppose $s\circ r^p$ is a rotation.\\
Then $s\circ r^p=r^k$ $(k\in Z)$.\\
Multiply both sides on the right by $r^{n-p}$\\
Then $s\circ r^p\circ r^{n-p}=r^k\circ r^{n-p}$\\
$\implies s=r^{k+n-p}$\\
This would mean $s$ is a rotation, which is a contradiction.\\
Therefore, $s\circ r^p$ must be a reflection.
\end{enumerate}

\noindent\textbf{Exercise \ref{exercise:symmetries:44}:}\\
%Using only associativity and Proposition~\ref{proposition:symmetries:Dn_generator_theorem}, complete the entire Cayley table for $D_4$.  Remember, there is a row and a column for each element of $D_4$. List the elements as indicated in Proposition~\ref{proposition:symmetries:D_elts}. You don't need to show all your computations. \emph{(But don't use tableau form--no cheating!)} 
	\begin{tabular}{c| c c c c c c c c }
		o & $id$ & $r$ & $r^2$ & $r^3$ & $s$ & $s\circ r$ & $s\circ r^2$ & $s\circ r^3$\\
		\hline
		$id$ & $id$ & $r$ & $r^2$ & $r^3$ & $s$ & $s\circ r$ & $s\circ r^2$ & $s\circ r^3$\\
		$r$ & $r$ & $r^2$ & $r^3$ & $id$ & $s\circ r^3$ & $s$ & $s\circ r$ & $s\circ r^2$\\
		$r^2$ & $r^2$ & $r^3$ & $id$ & $r$ & $s\circ r^2$ & $s\circ r^3$ & $s$ & $s\circ r$\\
		$r^3$ & $r^3$ & $id$ & $r$ & $r^2$ & $s\circ r$ & $s\circ r^2$ & $s\circ r^3$ & $s$\\
		$s$ & $s$ & $s\circ r$ & $s\circ r^2$ & $s\circ r^3$ & $id$ & $r$ & $r^2$ & $r^3$\\
		$s\circ r$ & $s\circ r$ & $s\circ r^2$ & $s\circ r^3$ & $s$ & $r^3$ & $id$ & $r$ & $r^2$\\
		$s\circ r^2$ & $s\circ r^2$ & $s\circ r^3$ & $s$ & $s\circ r$ & $r^2$ & $r^3$ & $id$ & $r$\\
		$s\circ r^3$ & $s\circ r^3$ & $s$ & $s\circ r$ & $s\circ r^2$ & $r$ & $r^2$ & $r^3$ & $id$
	\end{tabular}
\\

\noindent\textbf{Exercise \ref{exercise:symmetries:46}:}
\begin{enumerate}[{a.}]
\skipitems{1}

\item 
%Give the Cayley table for the four rotations of the square (4-sided polygon).  You may use $r$ to denote rotation by $90$ degrees, so that the rotations will be $\{ {\var id}, r, r^2, r^3  \}$.
\begin{tabular}{c| c c c c}
		o & $id$ & $r$ & $r^2$ & $r^3$\\
		\hline
		$id$ & $id$ & $r$ & $r^2$ & $r^3$\\
		$r$ & $r$ & $r^2$ & $r^3$ & $id$\\
		$r^2$ & $r^2$ & $r^3$ & $id$ & $r$\\
		$r^3$ & $r^3$ & $id$ & $r$ & $r^2$
	\end{tabular}
	
\item 
%Give the Cayley table for the four complex $4'{th}$ roots of unity. You may use $z$ to denote cis($\pi / 2$) so that the roots will be $\{ 1, z, z^2, z^3 \}$.
\begin{tabular}{c| c c c c}
		o & $1$ & $z$ & $z^2$ & $z^3$\\
		\hline
		$1$ & $1$ & $z$ & $z^2$ & $z^3$\\
		$z$ & $z$ & $z^2$ & $z^3$ & $1$\\
		$z^2$ & $z^2$ & $z^3$ & $1$ & $z$\\
		$z^3$ & $z^3$ & $1$ & $z$ & $z^2$
	\end{tabular}
\end{enumerate}
