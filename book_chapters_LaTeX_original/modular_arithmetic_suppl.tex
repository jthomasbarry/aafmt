\section{Solutions for ``Modular Arithmetic''}
\noindent \textbf{\textit{ (Chapter \ref{modular})}}\bigskip

\noindent\textbf{Exercise \ref{exercise:modular:racetrack_displacements}:}\\
a. Net displacement $=346-432+99=13$.\\
b. Net displacement $=44+13.53=733$.\\
\\
\textbf{Exercise \ref{exercise:modular:racetrack_positions}:}\\
a. Position: mod(13,5)=3.\\
   Position: mod(733,5)=3.\\
b. Both cases have the same position.\\
\\
\textbf{Exercise \ref{exercise:modular:equivdef}:}\\
by m, of b, by m, $r>s$, $a=pm+r$, $b=qm+s$, $a-b=(p-q)m+(r-s)$, between 0 and m, $r-s>0$, $r-s<m$, between 0 and m, by m.\\
\\
\textbf{Exercise \ref{exercise:modular:eqproof}:}\\
$a \equiv b$ (mod n) $\rightarrow n|(a-b)$ (Prop.\ref{proposition:modular:equivalence_alt})\\
$b \equiv c$ (mod n) $\rightarrow n|(b-c)$\\
$\rightarrow a-b=kn$ and $ b-c=ln$ (k,l are integers).\\
$\rightarrow a-c=(k+l)n \rightarrow n|(a-c) \implies a \equiv c$ (mod n).\\
\\
\textbf{Exercise \ref{exercise:modular:jan25}:}\\
a. 3 mod 7 = 3 and 25 mod 7 = 4 $\implies$  Jan 3 is not Thursday.\\
b. 7 doesn't divide $(31 - 25) \implies 31 \not\equiv 25$ (mod 7). So Jan 31 is not a Thursday.\\
\\
\textbf{Exercise \ref{exercise:modular:22}:}\\
c. $101 \equiv 29$ (mod 6): True\\
e. True\\
f. False. Can change it to $1476532 \equiv -71832772$ (mod 10)\\
\\
\textbf{Exercise \ref{exercise:modular:27}:}\\
$b \le n$, $-b \ge -n$, $a-b \ge -n$, $a \le n$ and $b \ge 0$, $-b \le 0$, $a-b \le n$\\
a - b is between -n and n, a-b is a multiple of n, n between -n and n is 0, a - b = 0, a = b\\
\\
\textbf{Exercise \ref{exercise:modular:28}:}\\
Example 3: Thursday (total 113 days from Dec 24, 2011 to Apr 15, 2012 and 113 mod 7 = 1).\\
Example 4: Feb 14, 2010 is Chinese New Year day. Feb 14, 2011 is 365 days after that.\\
365 mod 354 = 11 so Chinese New Year in 2011 is Feb 3.\\
Chinese New Year in 2012 is Jan 23 (354 days after Feb 3).\\
In 2009, Chinese New Year is Feb 25.\\
\\
\textbf{Exercise \ref{exercise:modular:UPCSymbols}:}\\
a. It's a valid UPC because $30 \equiv 0$ (mod 10).\\
b. Total sum is $33 \not\equiv 0$ (mod 10).\\
d. Transposition will be detected because the sum is $24 \not\equiv 0$ (mod 10).\\
Transposition error that can't be detected is 005000300426.\\
e. Error is detectable.\\
f. Suppose one code has the following digits: $d_1,d_2,d_3,...,d_{10}$ and the other code has the following digits: $e_1,e_2,e_3,...,e_{10}$ are both valid and suppose that the corresponding digits at each position are equal except for the $n^{th}$ digit $\implies d_n \neq e_n$ with $1 \le n \le 10$.\\
If n is even:\\
$(sum + d_n) \equiv (sum + e_n)$ (mod 10)\\
$\implies d_n \equiv e_n$ (mod 10) because $d_n, e_n \le 9$\\
$\implies d_n = e_n$\\
It contradicts to the original supposition.\\
If n is odd:\\
$(sum+3d_n) \equiv (sum+3e_n)$ (mod 10)\\
$\implies 10|(sum+3d_n-sum-3e_n)$\\
$\implies 10|3(d_n-e_n)$\\
Since $d_n-e_n \le 10$, the only solution to make 10 divides $3(d_n-e_n)$ is $3(d_n-e_n)=0 \implies d_n=e_n$.\\
It also contradicts to the original supposition.\\
Therefore UPC error detection scheme detects all single digit errors.\\
\\
\textbf{Exercise \ref{exercise:modular:ISBNCodes}:}\\
a. $3(1)+5(2)+4(3)+0(4)+9(5)+6(6)+0(7)+3(8)+5(9)+10(10) \equiv 0 \pmod{11}$\\
$\implies 275 \equiv 0 \pmod{11}$\\
c. The first 9 digits are arbitrary, and uniquely determine the 10th digit. So, $10^9$.\\
\\
\textbf{Exercise \ref{exercise:modular:mod_eq_1}:}\\
b. $25+x \equiv 6 \pmod{12}$\\
$1+x \equiv 6 \pmod{12}$\\
$x= 12k+6-1$\\
$x=5+12k$\\
\\
\textbf{Exercise \ref{exercise:modular:mod_eq_2}:}\\
g. $x=-1+7k$\\
i. $5x+1 \equiv 13 \pmod{26}$\\
$x= \displaystyle\frac{26k+12}{5}$\\
Let $k=3+5n$\\
$x=18+6n$\\
h. $5x+1 \equiv 13 \pmod{23}$\\
$x= \displaystyle\frac{23k+12}{5}$\\
Let $k=1+5n$\\
$x=7+23n$\\
\\
\textbf{Exercise \ref{exercise:modular:44}:}\\
a. Since $a \equiv b \pmod{n}$ and $c \equiv d \pmod{n}$\\
then $a=b+sn$ and $c=d+tn$\\
$\implies a.c =(b+sn)(d+tn)=bd+n(bt+sd+stn)$\\
By definition of $\odot$: $b \odot d \pmod{n} \to$ there is some integer q such that $bd=(b \odot d) +qn$\\
So we have: $ac=(b \odot d)+qn+n(bt+sd+stn)=b \odot d + n(q+bt+sd+stn)$\\
$\implies ac \equiv b \odot d \pmod{n}$\\
b. $x \ominus y=r iff x-y=r+sn$ and $r \in Z_n$\\
c. We have $a=b+sn$ and $c=d+tn$\\
$\implies a-c=b+sn-d-tn=b-d+n(s-t)$\\
By definition of $\ominus$: $b \ominus d \pmod{n} \to$ there is some integer k such that $b-d=b \ominus d +kn$\\
Therefore $a-c=b \ominus d +n(k+s-t) \implies a-c=b \ominus d \pmod{n}$\\
\\
\textbf{Exercise \ref{exercise:modular:ops}:}\\
a. $a+c \equiv b \oplus d \pmod{n}$ (Prop. 43)\\
Let call $a+c=k$ and $b \oplus d=r$\\
$\implies (a+c)+e=k+e$\\
$\implies (a+c)+e \equiv r \oplus f \pmod{n}$ (Prop. 43)\\
$\implies (a+c)+e \equiv (b \oplus d) \oplus f \pmod{n}$\\
b. $ac \equiv b \odot d \pmod{n}$ (Prop. 43)\\
Let $ac=k$ and $b \odot d =r$\\
$\implies k \equiv r \pmod{n}$\\
$\implies k+e \equiv r \oplus f \pmod{n}$ (Prop. 43)\\
$\implies ac+e \equiv (b \odot d) \oplus f \pmod{n}$\\
c. $a+c \equiv b \oplus d \pmod{n}$ (Prop. 43)\\
Let $a+c=k$ and $b \oplus d =r$\\
$k \equiv r \pmod{n}$\\
$ke \equiv r \odot f \pmod{n}$ (Prop. 43)\\
$\implies (a+c)e \equiv (b \oplus d) \odot f \pmod{n}$\\
\\

\textbf{Exercise \ref{exercise:modular:prove}:}\\
b. $(5 \cdot 9)+7 \equiv (5 \cdot 9)+6 \pmod{10}$\\
$\implies 2 \not\equiv 1 \pmod{10}$\\

\textbf{Exercise \ref{exercise:modular:49}:}\\
a. $\equiv 1 \pmod{5}$\\
b. $\equiv 2 \pmod{5}$\\
c. $\equiv 1 \pmod{5}$\\
d. $\equiv 0 \pmod{5}$\\
\\
\textbf{Exercise \ref{exercise:modular:52}:}\\
Suppose $a,b \in Z_n$ then by the definition of modular multiplication (Def. 42), we have $a \odot b =r \pmod{n}$ where $r \in \{0,1,...,n-1\}$\\
$\implies a \odot b \in Z_n$\\
$\implies Z_n$ is closed under modular multiplication.\\
\\
\textbf{Exercise \ref{exercise:modular:53}:}\\
b. Rational numbers: closed under addition, subtraction, multiplication.\\
d. Positive rational numbers: closed under addition and multiplication.\\
e. Positive real number: closed under addition, subtraction, multiplication, divison, and square root.\\
\\
\textbf{Exercise \ref{exercise:modular:54}:}\\
Let $z_1=a+bi$ and $z_2=c+di$, then:\\
$z_1z_2=(a+bi)(c+di)=(ac-bd)+(ad+bc)i$ is a complex number\\
$\implies$  Complex numbers are closed under multiplication.\\
\\
\textbf{Exercise \ref{exercise:modular:56}:}\\
Given any a $\in Z_n, (n>1)$ then $a \odot 1$ is computed as follows:\\
a. Compute $a.1$ using ordinary multiplication\\
b. taking the remainder mod n \\
Since $a1=a$, and $0 \le a <n$, it follows that the remainder is also a.\\
Hence $a \odot 1=a$.\\
Similarly, we can show that $1 \odot a=a$.\\
So, 1 satisfies the definition of multiplicative identity for $Z_n$ when $n>1$.\\
When $n=1$, $Z_1=\{0\} \to$ multiplicative identity for $Z_1$ is 0.\\
\\
\textbf{Exercise \ref{exercise:modular:58}:}\\
a. $0 \in Z_n$ and $0 \oplus 0=0 \to$ additive inverse of $0 \in Z_n$ is 0.\\
b. Suppose $a \in Z_n \setminus \{0\}$ and let $a'=n-a$\\
We have $0<a<n \implies n>a'>0 \implies a' \in Z_n$\\
We also have $a \oplus a'=a+n-a \pmod{n}=n \pmod{n}=0 \pmod{n}$\\
Similarly, we have $a' \oplus a=0 \pmod{n}$\\
$\implies a'$ is the additive inverse of a.\\
\\
\textbf{Exercise \ref{exercise:modular:60}:}\\
a. n = 3\\
b. n = 7\\
\\
\textbf{Exercise \ref{exercise:modular:64}:}\\
a. 0 has no multiplicative inverse under $\odot \implies Z_n$ is not a group.\\
b. $Z_3 \setminus \{0\}$ is a group under $\odot$.\\
\\
\textbf{Exercise \ref{exercise:modular:stick_units}:}\\
b. 1 ft\\
\\
\textbf{Exercise \ref{exercise:modular:m53}:}\\%name
b. 12 in\\
\\
\textbf{Exercise \ref{exercise:modular:gcd}:}\\
b. 11\\
\\
\textbf{Exercise \ref{exercise:modular:71}:}\\
a. We have $r_2=b-r_1q_2$ (Prop. 68)\\
We also have $r_1=a-bq_1$ (from previous step)\\
$\implies r_2=b-(a-bq_1)q_2=b(1+q_1q_2)-aq_2$ where $q_1$ and $q_2$ are integers.\\
$\implies r_2$ can also be written in the form $r_2=na+mb$ where n and m are integers.\\
b. For $k>2$, we have:\\
$r_{k-2}=na+mb$\\
$r_{k-1}=sa+tb$\\
We also have $r_k=r_{k-2}-r_{k-1}q_k$ (Prop. 68)\\
$\implies r_k=na+mb-(sa+tb)q_k=a(n-sq_k)+b(m-tq_k)$\\
$\implies r_k$ can also be written in the form $r_k=va+wb$ where $v$ and $w$ are integers.\\
c. From Prop. 68, $r_k$ is the gcd of two numbers a and b.\\
From b. we know that $r_k$ can be written in the form $na+mb$.\\
$\implies$  the gcd of two numbers a and b can be written in the form $na+mb$ where $n$ and $m$ are integers.\\
\\
\textbf{Exercise \ref{exercise:modular:dio1}:}\\
e. $m=7241k+1371$ $n=-3524k-151$\\
\\
\textbf{Exercise \ref{exercise:modular:diophant}:}\\
b. 3 and 27 are divisible by 3. 2 is not divisible by 3.\\
\\
\textbf{Exercise \ref{exercise:modular:m518}:}\\%name
a. $m=1590-119k$ $n=242k-3233$\\
b. $x=m$\\
c. $y=n$\\
\\
\textbf{Exercise \ref{exercise:modular:m519}:}\\%name
$am \equiv c \pmod{b}$\\
\\
\textbf{Exercise \ref{exercise:modular:m522}:}\\%name
No solution. The gcd of 504 and 1002 is 3. 919 is not divisible by 3. Prop. 3.5.20\\
\\
\textbf{Exercise \ref{exercise:modular:90}:}\\
a. $Z_5 \setminus \{0\}$ forms a group.\\
b. $Z_7 \setminus \{0\}$ forms a group.\\
c. $Z_9 \setminus \{0\}$ doesn't form a group because element 3 doesn't have multiplicative inverse.\\
d. $Z_n \setminus \{0\}$ with n is a prime number will form a group under $\odot$.\\
\\
\textbf{Exercise \ref{exercise:modular:93}:}\\
Let $a \in Z_n \setminus \{0\}$, we need to find $x$ is the multiplicative inverse of $a$\\
$\implies ax \equiv 1 \pmod{n}$.
According to Prop. 78, we can only find $x$ if 1 is an integer multiple of the gcd of $a$ and $n$\\
$\implies$  1 is the gcd of $a$ and $n$ in order to find $x$, which is the multiplicative inverse of $a$.\\
$\implies a$ is relatively prime to $n$ to make $x$ exist.\\
If $a$ is not relatively prime to $n \to$ they have a gcd which is bigger than 1 $\implies$  1 is not an integer multiple of the gcd of $a$ and $n \to$ we can't find $x$ (Prop. 78) $\implies a$ doesn't have an inverse under multiplication (mod n).\\
\\
\textbf{Exercise \ref{exercise:modular:94}:}\\
For $Z_n \setminus \{0\}$, if n is not prime, there is $1<a<n \in Z_n$ that $a|n$\\
$\implies$  the gcd of $a$ and $n$ is bigger than 1 $\implies a$ does not have a multiplicative inverse (according to Exercise 86) $\implies Z_n \setminus \{0\}$ is not a group under $\odot$.\\
\\
\textbf{Exercise \ref{exercise:modular:91}:}\\
a. $x=11 \pmod{12}$\\
c. $x=237 \pmod{403}$\\
\\
\textbf{Exercise \ref{exercise:modular:m534}:}\\%name
a. $x=59 \pmod{60}$\\
\\
