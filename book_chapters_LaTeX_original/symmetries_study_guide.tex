\section{Study guide  for ``Symmetries of Plane Figures''  chapter}\label{sec:Symmetry:study} 


\subsection*{Section \ref{definition_examples}, Definition and examples}
\subsubsection*{Concepts}
\begin{enumerate}
\item 
A symmetry of a geometrical object is a rearrangement (function) that preserves angles and distances. (Definition~\ref{symmDef})
\item
The identity symmetry (object remains fixed)
\item
Types of symmetries: rotations and reflections (``flips'')  (rotations are measured counterclockwise) (Example~\ref{example:Symmetry:mercedeslogo})
\item
Shifts(translation) are not symmetries (Example~\ref{example:Symmetry:mercedeslogo})
\item
Symmetries are bijections, but not all bijections are symmetries (Proposition~\ref{proposition:Symmetry:SymmBijection})
\end{enumerate}

\subsubsection*{Competencies}
\begin{enumerate}
\item
List the symmetries of an object. (\ref{exercise:Symmetry:mercedes})
\item
Find bijections that are not symmetries (\ref{exercise:Symmetry:bijectnotsym})
\item
Determine if a motion is a symmetry (\ref{exercise:Symmetry:10})
\item
Write the symmetries of a polygon as functions of its vertices (\ref{exercise:Symmetry:10})
\end{enumerate}


\subsection*{Section \ref{composition}, Composition of symmetries}
\subsubsection*{Concepts:}
\begin{enumerate}
\item
Horizontal (left-right) and vertical (up-down) reflections
\item
Symmetries may be expressed in various notations (ordered pair, tableau, arrow diagram, etc)
\item
Composition of symmetries work right to left.
\item
A composition of symmetries is also a symmetry. (Proposition~\ref{proposition:Symmetry:symcompclosed})
\item
Identity symmetry is denoted by $id$
\end{enumerate}

\subsubsection*{Competencies}
\begin{enumerate}
\item
Write the given function in a specified notation style. (Example~\ref{example:Symmetry:sym_tableau} and Exercises \ref{exercise:Symmetry:SymmComposition}, \ref{exercise:Symmetry:16})
\item
Compose a given set of symmetries. (Example~\ref{example:Symmetry:sym_comp_tableau} and Exercises \ref{exercise:Symmetry:SymmComposition}, \ref{exercise:Symmetry:CompSymm},  \ref{exercise:Symmetry:16})
\end{enumerate}


\subsection*{Section \ref{SymmetryGroup}, Do the symmetries of an object form a group}
\subsubsection*{Concepts:}
\begin{enumerate}
\item 
Cayley tables are used to represent composition operations. The table entry for $f \compose g$ is located in row $f$ and column $g$ (Table~\ref{S3_table})
\item
Know the properties of a group (closure, identity, associativity, inverse) and how they pertain to symmetries.  
\end{enumerate}

\subsubsection*{Competencies}
\begin{enumerate}
\item
Find entries in a Cayley table, by writing the symmetries in tableau form and then composing them. (Exercises~\ref{exercise:Symmetry:S3Table},\ref{exercise:Symmetry:symm_rectangle}, \ref{exercise:Symmetry:symm_square})
\item
Determine the inverses of symmetries. (\ref{exercise:Symmetry:symm_rectangle},  \ref{exercise:Symmetry:symm_square})
\item
Compare symmetry groups of different objects. (\ref{exercise:Symmetry:describesymm})
\end{enumerate}


\subsection*{Section \ref{sec:dihedral}, The dihedral groups}
\subsubsection*{Concepts:}
\begin{enumerate}
\item 
The $nth$ dihedral group, denoted $D_n$, is the group of symmetries of a regular $n$-gon.
\item
Reflections may be characterized by which vertices they leave fixed (Exercises~\ref{exercise:Symmetry:36},\ref{exercise:Symmetry:PentagonRefl}\ref{exercise:Symmetry:PentagonRefl},\ref{exercise:Symmetry:HexagonRefl},\ref{exercise:Symmetry:nonagon}) 
\item
$D_n$ is a group of order $2n$.  Each $D_n$ has $n$ rotations and $n$ reflections. (Proposition~\ref{proposition:Symmetry:dihReflRot})
\item
All reflections are of order 2: i.e. any reflection is the product of disjoint transpositions (Proposition~\ref{proposition:Symmetry:dihReflRot})

\item
$D_n$ is generated by one rotation and one reflection.  (Proposition~\ref{proposition:Symmetry:Dn_generator_theorem})
\end{enumerate}

\subsubsection*{Competencies}
\begin{enumerate}
\item
Write the reflections of an object in tableau form and determine how many vertices are fixed. (Exercises~\ref{exercise:Symmetry:reflection_pentagon}, \ref{exercise:Symmetry:31}, \ref{exercise:Symmetry:36} - \ref{exercise:Symmetry:HexagonRefl})
\item
Using  The composition relations between $r$ and $s$ given in  Proposition~\ref{proposition:Symmetry:Dn_generator_theorem}, compute Cayley tables for $D_n$, where the elements $D_n$ are specified as $s^k r^\ell$. (Exercises~\ref{exercise:Symmetry:44}, \ref{exercise:Symmetry:D5})
\end{enumerate}







