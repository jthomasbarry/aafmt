\section{Hints for ``Polynomial Rings'' exercises}\label{sec:polyrings:hints} 

\noindent Exercise \ref{exercise:poly:poly4}(d):  Note $4=3+1$ and $11=3^2+2$.

\noindent Exercise \ref{exercise:poly:multform}(d): Are there any common factors you can take outside 
the summations before multiplying?

\noindent Exercise \ref{exercise:poly:5Z}:  Note that if  $a_2$ is in $5\mathbb{Z}$, then $a_2 = 5a_2^{\prime}$ where $a_2^{\prime}$ is also an integer.
The same thing holds for al the other coefficients in $p(x)$ and $q(x)$.

\noindent Exercise \ref{exercise:poly:noInvFor0}:  Consider the expression $a\cdot (b + 0)$, where $a,b$ are arbitrary elements of $R$. Use the distributive law and add the additive inverse of  $a \cdot b$ to both sides.

\noindent Exercise \ref{exercise:poly:zerodivisor}:  Take the expression $ab=0$ and multiply both sides on the left by $a^{-1}$.

\noindent Exercise \ref{exercise:poly:zdzp}:  Use Exercise \ref{exercise:poly:zerodivisor} part b and remember that $\mathbb{Z}_p$ is a field.

\noindent Exercise \ref{exercise:poly:roots}(b):  Use the method in Section \ref{rootsingeneral}.
 