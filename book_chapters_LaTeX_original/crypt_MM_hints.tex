\section{Hints for ``Further  Topics in Cryptography'' exercises}\label{sec:further_crypt:hints}

\noindent Exercise \ref{exercise:further_crypt:dubroots} (a):  If there is a double root , then it must be that $0 = x^3 + ax + b$ has a double root. This means that the equation can be factored:  $x^3 + ax+b = (x-r_1)^2(x-r_2)$.  Express $a$ and $b$ in terms of $r_1$ and $r_2$.  (A similar approach can be used in the case of a triple root.)  

\noindent Exercise \ref{exercise:further_crypt:dubroots} (b):  There are 2 cases: (i) $b=0$, (ii) $b \neq 0$. In the case $b \neq 0$, first, show that $a > 0$.  Then use part (a) to express $r_1$ and $r_2$ in terms of $a$ and $b$, and show the equation factors properly.

